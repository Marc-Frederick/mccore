%%%%%%%%%%%%%%%%%%%%%%%%%%%%%% -*- Mode: Latex -*- %%%%%%%%%%%%%%%%%%%%%%%%%%%%
%% McCoreManual.tex
%% Copyright � 2001 Laboratoire de Biologie Informatique et Th�orique.
%%                  Universit� de Montr�al.
%% Author           : Martin Larose <larosem@iro.umontreal.ca>
%% Created On       : Mon Feb  5 18:54:17 2001
%% Last Modified By : Martin Larose
%% Last Modified On : Tue Aug 14 12:33:14 2001
%% Update Count     : 3
%% Status           : In development.
%%%%%%%%%%%%%%%%%%%%%%%%%%%%%%%%%%%%%%%%%%%%%%%%%%%%%%%%%%%%%%%%%%%%%%%%%%%%%%%


\documentclass[11pt]{article}
\usepackage{times,isolatin1,epsfig}
\usepackage[ps2pdf,backref,hyperindex,colorlinks,bookmarks]{hyperref}
\title{{\sc McCore} manual}
\author{Laboratoire de Biologie Informatique et Th�orique}
\date{\today}

\pagestyle{myheadings}
\markboth{McCore Manual}{McCore Manual}


\begin{document}
\maketitle

{\parindent 0pt \rule{\textwidth}{.5mm}}

\vspace{20pt}
\tableofcontents
\newpage

This manual is intended for internal use.  It describes the basic classes
and concepts needed for comprehending the general {\sc mcsym} packages.  The
library is named throughout this manual as {\sc mccore}.


\section{Atom types}

The abstract concept of atoms is represented as a hierarchy of classes.  The
base atom type {\tt t\_Atom} is an abstract class that declares methods for
identifying and saving the atom.  {\tt t\_Atom} is the type used within the
library whenever you want to declare an atom type.  

fichier de declaration
fichier de definition

residue types
fichier de declaration
fichier de definition

streams

Binstream

Pdbstream

compressed streams

CMessageQueue

exceptions

Points

Description d'un atome

filtres d'atomes

Residue id

Residue id set

Transfo

Description d'un r�sidu
composition
iterateurs
application d'un filtre
optention d'un residu par un iterateur

Model

Algo

Graph

HBond

McCore

\end{document}